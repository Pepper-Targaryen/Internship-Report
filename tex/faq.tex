%# -*- coding: utf-8-unix -*-
% !TEX program = xelatex
% !TEX root = ../thesis.tex
% !TEX encoding = UTF-8 Unicode

\chapter{Plot of Romance of the Three Kingdoms}

\section{Dong Zhuo's tyranny}
\label{sec:Dong Zhuo}

The missing emperor and the prince were found by soldiers of the warlord Dong Zhuo, who seized control of the imperial capital, Luoyang, under the pretext of protecting the emperor. Dong Zhuo later deposed Emperor Shao and replaced him with the Prince of Chenliu (Emperor Xian), who was merely a figurehead under his control. Dong Zhuo monopolised state power, persecuted his political opponents and oppressed the common people for his personal gain. There were two attempts on his life: the first was by a military officer, Wu Fu (伍孚), who failed and died a gruesome death; the second was by Cao Cao, whose attempt went awry and forced him to flee.

Cao Cao escaped from Luoyang, returned to his hometown and sent out a fake imperial edict to various regional officials and warlords, calling them to rise up against Dong Zhuo. Under Yuan Shao's leadership, 18 warlords formed a coalition army and launched a punitive campaign against Dong Zhuo. Dong Zhuo felt threatened after losing the battles of Sishui Pass and Hulao Pass, so he evacuated Luoyang and moved the imperial capital to Chang'an. He forced Luoyang's residents to move together with him and had the city set aflame. The coalition eventually broke up due to poor leadership and conflicting interests among its members. Meanwhile, in Chang'an, Dong Zhuo was betrayed and murdered by his foster son Lü Bu in a dispute over the maiden Diaochan as part of a plot orchestrated by the minister Wang Yun.

\section{Liu Bei's ambition}
\label{sec:Liu Bei}

Liu Bei and his oath brothers Guan Yu and Zhang Fei swore allegiance to the Han Empire in the Oath of the Peach Garden and pledged to do their best for the people. However, their ambitions were not realised as they did not receive due recognition for helping to suppress the Yellow Turban Rebellion and participating in the campaign against Dong Zhuo. After Liu Bei succeeded Tao Qian as the governor of Xu Province, he offered shelter to Lü Bu, who had just been defeated by Cao Cao. However, Lü Bu betrayed his host, seized control of the province and attacked Liu Bei. Liu Bei combined forces with Cao Cao and defeated Lü Bu at the Battle of Xiapi. Liu Bei then followed Cao Cao back to the imperial capital, Xu, where Emperor Xian honoured him as his "Imperial Uncle". When Cao Cao showed signs that he wanted to usurp the throne, Emperor Xian wrote a secret decree in blood to his father-in-law, Dong Cheng, and ordered him to get rid of Cao. Dong Cheng secretly contacted Liu Bei and others and they planned to assassinate Cao Cao. However, the plot was leaked out and Cao Cao had Dong Cheng and the others arrested and executed along with their families. Liu Bei eventually left Yuan Shao and established a new base in Runan, where he lost to Cao Cao again. He retreated south to Jing Province, where he found shelter under the governor, Liu Biao.

\section{Zhuge Liang's campaigns}
\label{sec:Zhuge Liang}

After Liu Bei's death, Cao Pi induced several forces, including Sun Quan, a turncoat Shu general Meng Da, the Nanman and Qiang tribes, to attack Shu, in coordination with a Wei army. However, Zhuge Liang managed to make the five armies retreat without any bloodshed. He also sent Deng Zhi to make peace with Sun Quan and restore the alliance between Shu and Wu. Zhuge Liang then personally led a southern campaign against the Nanman, defeated them seven times, and won the allegiance of the Nanman king, Meng Huo.

After pacifying the south, Zhuge Liang led the Shu army on five military expeditions to attack Wei as part of his mission to restore the Han dynasty. However, his days were numbered because he had been suffering from chronic illness and his condition worsened under stress. He would die of illness at the Battle of Wuzhang Plains while leading a stalemate battle against the Wei general Sima Yi.

% \chapter{常见问题}
% \label{chap:faq}

% {\bfseries{}Q:我是否能够自由使用这份模板?}

% A:这份模板以Apache License 2.0开源许可证发布,请遵循许可证规范。

% {\bfseries{}Q:我的论文是Word排版的,学校图书馆是不是只收 \LaTeX 排版的论文?}

% A:当然不是,Word版论文肯定收。

% {\bfseries{}Q:我的论文是 \LaTeX 排版的,学校图书馆是不是只收Word排版的论文?}

% A:当然不是,PDF版的电子论文是可以上交的。是否要交Word版就看你导师的喜好了。

% {\bfseries{}Q:为什么屏幕上显示的左右页边距不一样?}

% A:模板默认是双面打印,迎面页和背面页的页边距是要交换的,多出来的那一部分是留作装订的。

% {\bfseries{}Q:为什么在参考文献中会有“//”符号?}

% A:那就是国标GBT7714参考文献风格规定的。但可以使用 gbpunctin=false 选项将其还原成 in:,进一步可以在导言区加入\verb+\DefineBibliographyStrings{english}{in={}}+将其去掉。

% {\bfseries{}Q:为什么参考文献中会有[s.n.],[S.l], [EB/OL]等符号?}

% A: 那也是国标GBT7714参考文献风格定义的。[s.n.]表示出版者不祥,[S.l]表示出版地不祥,[EB/OL]表示引用的参考文献类型为在线电子文档。但可以使用gbpub=false 选项将其缺省补充的出版项[s.n.]等去掉。也可以使用选项 gbtype=false 将参考文献类型标识去掉。

% {\bfseries{}Q:如何获得帮助和反馈意见?}

% A:你可以通过\href{https://github.com/sjtug/SJTUThesis/issues}{在github上开issue}
% 、在\href{https://bbs.sjtu.edu.cn/bbsdoc?board=TeX_LaTeX}{水源LaTeX版}发帖反映你使用过程中遇到的问题。

% {\bfseries{}Q:使用文本编辑器查看tex文件时遇到乱码?}

% A:请确保你的文本编辑器使用UTF-8编码打开了tex源文件。

% {\bfseries{}Q:在CTeX编译模板遇到“rsfs10.tfm already exists”的错误提示?}

% A:请删除\verb+X:\CTEX\UserData\fonts\tfm\public\rsfs+下的文件再重新编译。问题讨论见\href{https://bbs.sjtu.edu.cn/bbstcon,board,TeX_LaTeX,reid,1352982719.html}{水源2023号帖}。

% {\bfseries{}Q:升级了TeXLive 2012,编译后的文档出现“minus”等字样?}

% A:这是xltxtra和fontspec宏包导致的问题。学位论文模板从0.5起使用metatlog宏包代替xltxtra生成 \XeTeX 标志,解决了这个问题。

% {\bfseries{}Q:为什么在bib中加入的参考文献,没有在参考文献列表中出现?}

% A: bib中的参考文献条目,常通过\verb+\cite+或\verb+\parencite+或\verb+\supercite+或\verb+\textcite+等命令在正文中引用进而加入到参考文献列表中。当需要将参考文献条目加入到文献表中但又不引用可以使用\verb+\nocite+命令,当nocite参数为*时则引入bib中的所有文献。
% %\verb+\upcite+ 是哪个宏包的?之前没有见过

% {\bfseries{}Q:我可以使用Sublime Text编写学位论文吗?}

% A: 可以。首先\href{https://www.sublimetext.com/}{下载}并安装Sublime Text,然后安装
% \href{https://packagecontrol.io/installation}{Package Control},
% 之后按\verb|ctrl+shift+p|或者\verb|cmd+shift+p|调出命令窗口,
% 输入\verb|install|,选择\textit{Package Control: Install Package},按回车,
% 稍等片刻,等待索引载入后会弹出选项框,输入\verb|LaTeXTools|并回车,即可成功安装插件。
% 之后只需要打开\verb|.tex|文件,按\verb|ctrl+b|或者\verb|cmd+b|即可编译,
% 如有错误,双击错误信息可以跳转到出错的行。

% {\bfseries{}Q:在macTex中,为什么pdf图片无法插入?}

% A:如果报错是“pdf: image inclusion failed for "./figure/chap2/sjtulogo.pdf".”,则采取以下步骤

% \begin{lstlisting}[basicstyle=\small\ttfamily, caption={编译模板}, numbers=none]
% brew install xpdf
% wget http://mirrors.ctan.org/support/epstopdf.zip
% unzip epstopdf.zip
% cp epstopdf/epstopdf.pl /usr/local/bin/
% cd figure/chap2
% pdftops sjtulogo.pdf
% epstopdf sjtulogo.ps
% pdfcrop sjtulogo.pdf
% mv sjtulogo.pdf backup.pdf
% mv sjtulogo-crop.pdf sjtulogo.pdf
% \end{lstlisting}

% {\bfseries{}Q:为什么维普等查重系统无法识别此模板生成的 pdf 内所有的中文?}

% A: 中文无法识别的情况多半是由于使用了 ShareLaTeX 的原因,请尝试使用 TexStudio 等软件在本地进行编译。
% 如果使用 TeXstudio 请在 Preferences-Build 中将 Default Compiler 和 Default Bibliography Tool 分别改为 XeLaTeX 和 Biber。

% {\bfseries{}Q:如何向你致谢?}

% A: 烦请在模板的\href{https://github.com/sjtug/SJTUThesis}{github主页}点击“Star”,我想粗略统计一下使用学位论文模板的人数,谢谢大家。非常欢迎大家向项目贡献代码。
