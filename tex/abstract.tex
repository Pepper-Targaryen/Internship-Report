%# -*- coding: utf-8-unix -*-
% !TEX program = xelatex
% !TEX root = ../thesis.tex
% !TEX encoding = UTF-8 Unicode
%%==================================================
%% abstract.tex for SJTU Master Thesis
%%==================================================

\begin{abstract}

    \hspace{2em}中文摘要需要两页,请注意中文大段落的缩进!

    \hspace{2em}《三国演义》,全名为《三国志通俗演义》,又称作《三国志传》、《三国全传》、《三国英雄志传》,是中国第一部长篇历史章回小说。作者一般被认为是元末明初的罗贯中。虚实结合,曲尽其妙,是四大名著中唯一根据历史事实(《三国志》)改编之小说,但也让许多人误以为《三国演义》的内容就是中国三国时期的正史。明末清初文学家、戏曲家李渔有言曰:“演义一书之奇,足以使学士读之而快,委巷不学之人读之而亦快;英雄豪杰读之而快,凡夫俗子读之而亦快。”

    \hspace{2em}《三国演义》描写的是从东汉末年到西晋初年近百年间的历史,反映了三国时代的政治军事斗争以及各类社会矛盾的渗透与转化。在对三国态度上,尊刘反曹鄙吴是民间的主要倾向,以刘备集团作为描写的中心,隐含着人民对汉室复兴的希望和皇室正统思想,尽管这些旧有观点已不容于今日。清人毛氏父子批改三国演义时,把明代流传下来的版本中不实讥望、怪力乱神之处删除勘正。鲁迅在《中国小说的历史的变迁》称:“因为三国的事情,不像五代那样纷乱;又不像楚汉那样简单;恰是不简不繁,适于作小说。而且三国时的英雄,智术武勇,非常动人,所以人都喜欢取来作小说的材料。”

    \hspace{2em}《三国演义》精于塑造人物形象,能通过叙述人物的行动和举止来反映人物独特个性。如诸葛亮、刘备、曹操、关羽、张飞等早已深入民心,性格突出,全赖罗贯中绘形绘声和生动逼真的描写,令读者如见其人,如闻其声。结构宏伟紧密,全书人物众多,头绪纷繁,情节复杂,惟作者能以蜀汉为中心,抓紧魏蜀吴三国间的冲突矛盾,写来井井有条,脉络清晰,规模宏大。语言方面夹用文言白话,明快生动,既吸收古代文言的精华,亦加以适当的通俗化,故能“文不甚深,言不甚俗”,收雅俗共赏之艺术效果。

    \hspace{2em}《三国演义》,全名为《三国志通俗演义》,又称作《三国志传》、《三国全传》、《三国英雄志传》,是中国第一部长篇历史章回小说。作者一般被认为是元末明初的罗贯中。虚实结合,曲尽其妙,是四大名著中唯一根据历史事实(《三国志》)改编之小说,但也让许多人误以为《三国演义》的内容就是中国三国时期的正史。明末清初文学家、戏曲家李渔有言曰:“演义一书之奇,足以使学士读之而快,委巷不学之人读之而亦快;英雄豪杰读之而快,凡夫俗子读之而亦快。”

    \hspace{2em}《三国演义》描写的是从东汉末年到西晋初年近百年间的历史,反映了三国时代的政治军事斗争以及各类社会矛盾的渗透与转化。在对三国态度上,尊刘反曹鄙吴是民间的主要倾向,以刘备集团作为描写的中心,隐含着人民对汉室复兴的希望和皇室正统思想,尽管这些旧有观点已不容于今日。清人毛氏父子批改三国演义时,把明代流传下来的版本中不实讥望、怪力乱神之处删除勘正。鲁迅在《中国小说的历史的变迁》称:“因为三国的事情,不像五代那样纷乱;又不像楚汉那样简单;恰是不简不繁,适于作小说。而且三国时的英雄,智术武勇,非常动人,所以人都喜欢取来作小说的材料。”

    \hspace{2em}《三国演义》精于塑造人物形象,能通过叙述人物的行动和举止来反映人物独特个性。如诸葛亮、刘备、曹操、关羽、张飞等早已深入民心,性格突出,全赖罗贯中绘形绘声和生动逼真的描写,令读者如见其人,如闻其声。结构宏伟紧密,全书人物众多,头绪纷繁,情节复杂,惟作者能以蜀汉为中心,抓紧魏蜀吴三国间的冲突矛盾,写来井井有条,脉络清晰,规模宏大。语言方面夹用文言白话,明快生动,既吸收古代文言的精华,亦加以适当的通俗化,故能“文不甚深,言不甚俗”,收雅俗共赏之艺术效果。
    
\end{abstract}

\begin{englishabstract}

    One-page english abstract.
    
    Romance of the Three Kingdoms is a 14th-century historical novel attributed to Luo Guanzhong. It is set in the turbulent years towards the end of the Han dynasty and the Three Kingdoms period in Chinese history, starting in 169 AD and ending with the reunification of the land in 280.
    
    The story – part historical, part legend, and part mythical – romanticises and dramatises the lives of feudal lords and their retainers, who tried to replace the dwindling Han dynasty or restore it. While the novel follows hundreds of characters, the focus is mainly on the three power blocs that emerged from the remnants of the Han dynasty, and would eventually form the three states of Cao Wei, Shu Han, and Eastern Wu. The novel deals with the plots, personal and military battles, intrigues, and struggles of these states to achieve dominance for almost 100 years.
    
    Romance of the Three Kingdoms is acclaimed as one of the Four Great Classical Novels of Chinese literature; it has a total of 800,000 words and nearly a thousand dramatic characters (mostly historical) in 120 chapters. The novel is among the most beloved works of literature in East Asia, and its literary influence in the region has been compared to that of the works of Shakespeare on English literature. It is arguably the most widely read historical novel in late imperial and modern China.

\end{englishabstract}

